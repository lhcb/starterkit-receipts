%!TEX TS-program = xelatex
%!TEX encoding = UTF-8 Unicode

\documentclass[$fontsize$, a4paper]{article}

% LAYOUT
%--------------------------------
\usepackage{geometry} 
\geometry{$geometry$}

% No page numbers
\pagenumbering{gobble}

% Left align
\usepackage[document]{ragged2e}

$if(letterhead)$
  % To include the letterhead
  \usepackage{wallpaper}
  \ULCornerWallPaper{1}{letterhead.pdf}
$endif$

% TYPOGRAPHY
%--------------------------------
\usepackage{fontspec} 
\usepackage{xunicode}
\usepackage{xltxtra}

% converts LaTeX specials (quotes, dashes etc.) to Unicode
\defaultfontfeatures{Mapping=tex-text}
\setromanfont [Ligatures={Common}, Numbers={OldStyle}]{$seriffont$}
\setsansfont[Scale=0.9]{$sansfont$}

% Set paragraph break
\setlength{\parskip}{1em}

% Custom ampersand
\newcommand{\amper}{{\fontspec[Scale=.95]{$seriffont$}\selectfont\itshape\&}}

$if(seriffont)$
\setmainfont[SmallCapsFeatures={LetterSpace=5,Letters=SmallCaps}]{$seriffont$}
$endif$
$if(sansfont)$
    \setsansfont{$sansfont$}
$endif$

% Command required by how Pandoc handles the list conversion
\providecommand{\tightlist}{%
  \setlength{\itemsep}{0pt}\setlength{\parskip}{0pt}}

% TABLE CUSTOMIZATION
%--------------------------------
\usepackage{spreadtab}
\usepackage[compact]{titlesec} % For customizing title sections
\titlespacing*{\section}{0pt}{3pt}{-7pt} % Remove margin bottom from the title
\usepackage{arydshln} % For the dotted line on the table
\renewcommand{\arraystretch}{1.5} % Apply vertical padding to table cells
\usepackage{hhline} % For single-cell borders
\usepackage{enumitem} % For customizing lists
\setlist{nolistsep} % No whitespace around list items
\setlist[itemize]{leftmargin=0.5cm} % Reduce list left indent
\setlength{\tabcolsep}{9pt} % Larger gutter between columns


% LANGUAGE
%--------------------------------
$if(lang)$
\usepackage{polyglossia}
\setmainlanguage{$lang$}
$endif$

% PDF SETUP
%--------------------------------
\usepackage[xetex, bookmarks, colorlinks, breaklinks]{hyperref}
\hypersetup
{
  pdfauthor=$author$,
  pdfsubject=Invoice Nr. $invoice-nr$,
  pdftitle=Invoice Nr. $invoice-nr$,
  linkcolor=blue,
  citecolor=blue,
  filecolor=black,
  urlcolor=blue
}

% To display custom date
% \usepackage[nodayofweek]{datetime}
% \newdate{date}{01}{12}{1867}
% \date{\displaydate{date}}
% Use this instead of \today: % \displaydate{date}

% DOCUMENT
%--------------------------------
\begin{document}
% Preamble
\newcounter{pos}

% Registrants
$for(registrants)$
\rmfamily
\begin{center}
    \small
    $if(logo)$\includegraphics[width=3cm]{$logo$}\\$endif$
    \textsc{\textbf{\MakeLowercase{$author$}}}
\end{center}

\vspace{5em}

\normalsize \sffamily
\textit{$registrants.name$}\\
$registrants.institute$\\
\texttt{$registrants.email$}\\

\vspace{6em}

\begin{flushright}
  \small
  $city$, \today
\end{flushright}

\vspace{1em}


%\section*{\textsc{receipt} \#$receipt-base$-$registrants.id$}
\section*{\textsc{receipt}}
\footnotesize
\setcounter{pos}{0}
\STautoround*{2} % Get spreadtab to always display the decimal part
$if(commasep)$\STsetdecimalsep{,}$endif$ % Use comma as decimal separator

\begin{spreadtab}{{tabular}[t t t]{lp{8.2cm}r}}
  \hdashline[1pt/1pt]
  @ \noalign{\vskip 2mm} \textbf{Pos.} & @ \textbf{Description} & @ \textbf{Prices in $currency$} \\ \hline
      $for(service)$ @ \noalign{\vskip 2mm} \refstepcounter{pos} \thepos 
        & @ $service.description$ 
        $if(service.details)$\newline {\rmfamily\scriptsize $service.details$} $endif$
        & $service.price$\\$endfor$ \noalign{\vskip 2mm} \hline
  @ & @ \multicolumn{1}{r}{Total:}                & :={sum(c1:[0,-1])} \\ \hhline{~~-}
\end{spreadtab}

\vspace{15mm}

\sffamily
\small

\medskip

\rmfamily
$separator$\pagebreak
$endfor$



\end{document}
